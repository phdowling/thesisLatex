\chapter{Related Work}\label{relatedwork}

Through the high risk potential in the medical field the requirements for simulators are particularly high. To assure that the simulation can actually benefit or even partly replace the traditional surgical education every part of the development and testing process has to be thoroughly documented and analysed. Therefore there are several fields of research to consider when aspiring to create a simulation device which adheres to the current standards. We will discuss some of these concepts specifically in the following. 

\section{Current State}


While simulation is supposedly beneficial for the education and assessment of all surgical procedures, minimally invasive procedures, such as interventional radiology or cardiology are particularly well fitted to be simulated, because the performance in these operations is eminently influenced by the surgeon’s skill to interpret a two-dimensional image, e.g. fluoroscopy or ultrasound.\parencite{johnson_virtual_2012}\parencite{green_current_2014} These procedures can be modeled through a combination of 2D visual representation and haptic feedback, unlike open surgery, which is more challenging due to its 3D work environment\parencite{pandey_expanding_2012}. Moreover, many movements in these processes are hard to entirely comprehend, as they are only perceptible through the imaging and the outside manipulation of the instruments by the surgeon. However, the main clue to reenact the movements, the haptic feedback, is only accessible to the trainee in practice. To minimize the potential harm for patients, simulation can offer the experience of these tactile responses without risking harm on real-life patients\parencite{johnson_development_2011}.
